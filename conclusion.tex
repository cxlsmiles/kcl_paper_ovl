\section{Conclusion}\label{sec:concl}

Using a combination of x-ray absorption and Auger electron spectroscopy, in this work we study the electronic structure of aqueous solution of KCl at the K-edges of both K and Cl.

First of all, the combination of the two experimental techniques allowed us to determine the ionization potentials of both aqueous K$^{+}$ and Cl$^{-}$ to be 3611.9\,eV and 2825.4\,eV, respectively, as well as the natural linewidths of the core ionized states -- 0.72\,eV in the case of K$^{+}$ and 0.62\,eV in the case of Cl$^{-}$.

Moreover, by analyzing the dispersive resonant Auger features on the Auger electron spectrum as a function of photon energy, we could determine the positions of the core excited states of aqueous \ki~and \cli -- at 3611.5\,eV and 2825.2\,eV, respectively. With the aid of ab initio calculations on small ion-water clusters with varying number of water molecules, we assign the core excited states to the have a predominantly 1s$\,\rightarrow\,$4p character. These cannot be observed in a pure XAS spectrum of the ions because due to the lifetime broadening these peaks overlap with the edge peak. Additionally, we see an isolated dispersive feature in the combined ... spectrum of \ki$_\text{aq}$. Using the calculations we attribute this feature to the resonant Auger decay of the dipole-forbidden 1s$\,\rightarrow\,$3d state which acquires intensity in a solution due to the lower symmetry of the solvent molecules around the ion, and due to its energetic proximity to the 1s$\,\rightarrow\,$3d state. Such a dispersive feature is absent in the XAS/AES spectrum of \cli. First of all the 1s$\,\rightarrow\,$4p and 1s$\,\rightarrow\,$3d states are far in energy and thus they mix weakly upon addition of water molecules. Second of all, the final Auger states populated in the decay of aqueous \ki~and \cli. The Auger electron spectrum carries information about the final Auger states


Finally, we would like to stress the importance of the combination of the two experimental techniques