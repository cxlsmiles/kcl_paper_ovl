In this work we investigate aqueous potassium chloride by a combination of near-edge X-ray absorption and Auger electron spectroscopy. The Auger electron spectra of the isoelectronic K+ and Cl- as a function of photon energy at the K-edges of both ions exhibit different features. With the aid of {\it ab initio} calculations of the core excited states as well as the final Auger states in the bare ions and in their microsolvated clusters we give an explanation of the observed differences. In the case of K+, the energetic proximity of the 1s$\,\rightarrow\,$4p and the dipole-forbidden 1s$\,\rightarrow\,$3d K-shell excitations results in the latter acquiring intensity in a solution where the symmetry of the ion is lowered. As a result, together with the pure spectator Auger decay, leading to the population of 2p-2 4p states, 2p-2 3d states are also populated, appearing as a separate island in the Auger electron spectrum. On the contrary, the 1s$\,\rightarrow\,$4p and 1s$\,\rightarrow\,$3d core excitations in \cli~are well separated in and they mix weakly upon solvation. Thus, the additional feature ... is missing in the combined XAS/AES spectrum of this ion. Our work shows that XAS/AES is a powerful tool for studying the electronic structure of ions in a solution.