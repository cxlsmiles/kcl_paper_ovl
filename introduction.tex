\section{Introduction}
% Paragraph 1: introduction to AES and XAS/AES and what are they needed for. if possible, find a combination of the two to explan that this is a more powerful tool to study the electronic structure

Understanding how solvated ions respond to irradiation with x-rays can give insight into the structure of solutions \citep{Pokapanich09:7264}, and also into the mechanism of radiation damage \citep{Stumpf16:237}. Upon absorption of an x-ray photon by a solvated metal ion, a core excited or core ionized state is populated depending on the photon energy. Subsequently, a series of ultrafast electronic decay processes such as Auger decay, interatomic Coulombic decay (ICD) and electron transfer mediated decay (ETMD), is initiated \citep{Pokapanich09:7264,Pokapanich11:13430,Stumpf16:237,unger17:708}. The course of a decay cascade depends on the character of the initially populated states. In the case of a core ionized state, the normal Auger decay process leads to the population of doubly ionized final states, which can be either localized on the initially ionized unit, or delocalized due to charge-transfer-to-solvent (REFS)\citep{ceolin17} or in a core-like ICD process (REFS)\citep{Pokapanich09:7264,Pokapanich11:13430}. In the case of a core excited state, the resonant Auger process competes with the process of delocalization of the excited electron. If the initially core excited electron delocalizes within the lifetime of the core hole, then normal instead of resonant Auger decay is observed \citep{ottosson12:1}\citep{Nordlund07:217406}(REFS). As a result, doubly ionized states are populated in contrast to the resonant Auger case, where the final states are doubly ionized excited. The process of delocalization in aqueous solutions is very fast, in the case of water, it occurs on a sub-femtosecond to femtosecond time scale depending on the photonenergy, and its rate can be controlled through the charge of the solvated ions \citep{Nordlund07:217406,Ottosson11:13489}. One of the aims of this study is to elucidate the nature of the states populated upon x-ray irradiation of solvated ions, and to understand whether the rate of delocalization influences the resonant Auger process.


Moreover, a decay cascade in a solvated ion is quite different from that in a bare ion due to the presence of the solvent molecules. The latter have two effects -- first, they affect the excited \citep{miteva16:16671} or ionized states of the ion and second, they can also participate in the decay processes \citep{Pokapanich09:7264,Stumpf16:237}, leading to the population of delocalized final states, and thus to a damage of the surrounding environment. Another aim of our study is thus to extract information about how the electronic structure of the solvated ions is influenced by the presence of the solvent and to relate this change to the course of the Auger processes.


In this work, we used Auger electron spectroscopy together with x-ray absorption spectroscopy to study the electronic structure of aqueous potassium chloride. In particular, we demonstrate experimentally that at photon energies below the K-edges of \ki~and \cli, localized core excited states of the two ions are populated. These states undergo resonant Auger decay within less than 1\,fs, thus outpacing the delocalization of the excited electron. Moreover, we observe that the although the \ki~and \cli~ions are isoelectronic, they have different fingerprints in the resonant Auger spectra. With the aid of high-level {\it ab initio} calculations of the initial and final states of the resonant Auger process of both the bare ions and their microsolvated clusters, we demonstrate that these differences are a result of the different electronic structure of the two ions, thus confirming that the XAS/AES technique is a sensitive probe of the electronic structure of solutions.


%
% 
%Consequently, as a first step to understand how the electronic properties of ions is influenced by the presence of a solvent / or how ion-water interaction influences the electronic structure of the ion
%
%outline briefly other spectroscopic techniques (\citep{ottosson12:1})
%\citep{ottosson12:1}
%
%. 
%
%\citep{Nordlund07:217406} --> resonant Auger vs normal Auger decay -- delocalization of the core-excited electron; at the pre-edge happens within more than 20\,fs, thus it would be slower than RA decay.
%
%in the Auger decay also species in the environment can be involved -- ICD \citep{Pokapanich09:7264}; appearance of delocalized states in the Auger spectrum.
%
%
%\citep{Ottosson11:13489} --> electron delocalization upon O1s core excitation in water and aqueous solutions of LiBr and MgBr2 slows down in electrolyte solutions and also its rate shows a dependence on the cationic charge; control of the rate of delocalization by using different cations; electrostatic interaction with the solvated metal ion leads to the back-polarization of the excited electron 
%
%
%using the core-hole clock method \citep{}, one can determine the lifetimes of different processes
%
%theoretically, this is a tedious task since it requires a combination of different methods, such as electronic structure, electron dynamics and molecular dynamics methods (no idea!). experimentally, one can use XAS spectroscopy .... spectroscopies. Used alone, however, these experimental techniques do not always provide complete information about the nature of core-excited states. The latter have short lifetimes (order of a few fs) and therefore broadening, the peaks overlap and become indistinguishable. However, combinations of spectroscopic methods, such as XAS/RIXS or XAS/AES give a complete picture of the processes and the electronic states populated in these decay processes. The aim of our work is to study the first step of x-ray initiated the electronic decay cascade of aqueous KCL
%Thus, the main objective of this study is to collect information such as binding energy position, intensity, configuration, on the core-excited states involving a transition of the 1s orbital to either the first empty orbitals or into the ionization continuum, and to characterize the relaxation mechanisms, based on the electronic processes we highlighted for their isoelectronic counterpart in the gas phase. Potential delocalized electronic states between the solute (K$^{+}$/Cl$^{-}$) and the solvent (water) will also be probed.
%
%% Paragraph 3: present what we did - experiment and calculations and outline the main results
%
%To this end we used a combination of the experimental XAS/AES technique and {\it ab initio} calculations on KCl.
%
%Our methods of investigation are the X-ray photoelectron spectroscopy and the (resonant) Auger spectroscopy in the hard X-ray range, and considering the high kinetic energy of the emitted electrons, this study is the first – to our knowledge – to clearly probe the bulk of the liquid, contrary to previous similar studies focusing on the liquid/vacuum interface.
% 
%The combination of XAS and AES spectroscopies is a powerful tool :D rich in information, it gives access to the ionization threshold of the solvated ions, the positions of the core-excited states from the resonant Auger dispersive features, positions of the final states, charge-transfer-to-solvent effects. To analyze the character of the core-excited states and the final Auger states we present {\it ab initio} calculations of the XAS spectra of microsolvated K$^{+}$(H$_2$O)$_n$ ($n = 1, 2, 4, 6$) clusters in order to interpret the experimental XAS and AES spectra. The calculations give further insight into the experimental results.
%
%% Paragraph 4: outline of the paper
%
%The paper is organized as follows. In the next section we present the experimental set-up and we outline the theoretical calculations. Then we show the experimental results followed by detailed discussion based on the theoretical calculations.
%
%%for example, IP decreases faster than the energy of the 1st core-excited resonance
%
%%term value = IP - $E_{1st\,resonance}$ $\rightarrow$ decreases as the number of water molecules increases
%
%
%
%% Paragraph 2: what happens in a solution - delocalization of the core excited state, ``competition'' between resonant and normal Auger, and what is the aim of the paper - still not clear