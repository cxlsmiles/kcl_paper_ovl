\section{Methods} \label{sec:methods}
\subsection{Experimental}
% Denis, please have a look at the experimental part and see if all the references have been correctly included

For the present experiment we used the newly operational microjet setup that was specifically designed for the HAXPES station of the GALAXIES beamline, in collaboration with the Microliquids Company. Details of the beamline are given in the reference [JPR] and details on the electron spectrometer are given in the reference \citep{ceolin15:022502,ceolin13:188}. Due to the limited amount of space available if front of the analyzer lens and due to the limited number of ports available on the main vacuum chamber, the challenge was to arrange all the elements composing the microjet setup in a very compact manner. A differentially-pumped tube in which the microjet assembly is inserted, is installed in front of the spectrometer lens and can be moved independently from the main vacuum chamber by means of a three axes motorized manipulator. Two holes of 2 mm diameter allow the photons to go in and out of the tube, and a third one, on which is mounted a skimmer having a 500$\mu$m diameter hole, allows the electrons created at the interaction point to go in the direction of the spectrometer lens. To ensure a proper vacuum in the differentially pumped tube, a 5-way cross is connected to it and holds a 1200 l/s turbo pump, a liquid nitrogen trap, a vacuum gage and the liquid/electrical feedthroughs. A rail, fixed at the bottom of the tube, is used to slides precisely an insert on which the different elements necessary for the injection, control, collection and visualization of the liquid are mounted.


The head of this insert is mostly composed by: a glass capillary fixed by a dedicated peek-piece, a catcher in CuBe with a 300$\mu$m hole, a piezo motors stage allowing a precise control of these last two parts, and a camera. The catcher is placed at a distance of about 5mm from the capillary and is permanently pumped in order to extract the liquid from the vacuum chamber. For the present experiment, a 0.5M KCl aqueous solution is injected in a capillary having a 30$\mu$m diameter, by a HPLC pump with a constant flux of 1.6 ml/min. The catcher temperature is controlled so that the liquid does not freeze before its extraction. Considering that the photon propagation axis, the spectrometer lens axis and the liquid microjet form an orthogonal trihedron, the piezo motorization allows moving the liquid microjet along the photon propagation and the lens axes during the experiment if necessary. The catcher can be moved by the piezo motors along the photon propagation axis only. A simple camera is used to control the liquid jet positioning as compared with the catcher hole. The alignment of the full head (particularly capillary plus catcher) as compared with the X-ray beam is performed by moving the whole system using the 3-axes motorized manipulator. The alignment of the setup is performed by measuring the O1s XPS peak intensity of a simple salt aqueous solution, and by optimizing the liquid phase vs gas phase ratio. The pressure in the main chamber was kept below the 10$^{-5}$ mbar range whereas it was kept at about 10$^{-4}$ mbar in the differentially pumped tube when the HPLC pump was ON. Our equipment is an updated version of the equipment used in the references [3-5]. The aqueous potassium chloride solution was prepared by mixing >99\% KCl salt with deionized water. Filtering and degazing procedures were systematically performed before injecting the solution into the microjet by the HPLC pump.


\subsection{{\bf{\it Ab initio}} calculations}

The X-ray absorption spectra of the microsolvated clusters of potassium and chlorine were computed at the ground state equilibrium geometries of K$^+$(H$_2$O)$_m$ and \cli(H$_2$O)$_m$, where m = 1, 2, 4, 6. All structures were optimized at the MP2 level of theory using the 6-311++G(2d,2p) basis set \citep{Krishnan80:650,Blaudeau97:5016}. A frequency analysis was carried out in order to confirm that the obtained geometries are minima on the respective potential energy surfaces. The geometry optimization and frequency analysis were performed with the Gaussian 09 package \citep{g09}. In the case of K$^{+}$(H$_2$O)$_6$ the ground state geometry was obtained by constrained geometry optimization starting with the equilibrium geometry \citep{lee99:3995} belonging to the D$_3$ point group and increasing the angle $\theta$ between the K-O bond and the $C_3$ axis to 55$^{\circ}$. This angle was chosen to be around the maximum in the O-K-O angular distribution obtained from quantum mechanics / molecular mechanics dynamical simulations in Ref.\ \citep{ma14:1006}. {\color{red} the maximum is at 70$^{\circ}$, in our calculation the angle is 90$^{\circ}$} The ground state equilibrium structures are presented in Figs.\ \ref{fg:knw_xas} and \ref{fg:clnw_xas}. In the case of potassium, they belong to the C$_{2\text{v}}$ (K$^{+}$(H$_2$O)), D$_{2\text{d}}$ (K$^{+}$(H$_2$O)$_2$), S$_4$ (K$^{+}$(H$_2$O)$_4$), and D$_3$ (K$^{+}$(H$_2$O)$_6$) point groups \citep{rao08:12944}. The optimized K-O distances are between 2.638 -- 2.842~\AA, which is in good agreement with other theoretical and experimental works \citep{Ohtaki93:1157,soper06:180,ma14:1006}. In the case of chlorine, the ground state equilibrium 1-, 2- and 6-coordinated structures belong to the C$_{1}$ point group, whereas the 4-coordinated structure belongs to the C$_4$ point group \citep{gora00:7}. The optimized Cl-O distances are between 3.093 -- {\color{red}3.305}~\AA, which is in good agreement with other theoretical and experimental works \citep{ge13:13169,gora00:7,Ohtaki93:1157,soper06:180,ma14:1006}. The 6-coordinated clusters can be considered as representatives of the complete first solvation shell of the two ions \citep{Ohtaki93:1157,soper06:180,ma14:1006}.


The energies and transition moments of the core excited states of the microsolvated clusters were computed with the algebraic diagrammatic construction method for the polarization propagator \citep{sch82:2395} within the core-valence separation approximation \citep{bar85:867,ced80:206,ced81:1038} (CVS-ADC(2)x) as implemented in the Q-Chem package \citep{Wenzel14:1900,Wenzel14:4583,Wormit14:774,QChem2015}. In the case of \cli, the 6-311++G(3df,3pd) basis set \citep{Krishnan80:650,McLean80:5639} (excluding the f functions) was used on all atoms, whereas in the case of \ki, we used the 6-311+G(2d,p) basis set \citep{Krishnan80:650,Blaudeau97:5016} on all atoms, and an additional set of 2s, 2p and 2d diffuse functions was added on K. In our calculations, the core space comprises the 1s orbital of K$^{+}$ or \cli, whereas the remaining occupied orbitals are included in the valence space. For the calculations of the XAS spectra we used the D$_2$ point group in the case of K$^{+}$(H$_2$O)$_2$, the C$_2$ point group in the case of K$^{+}$(H$_2$O)$_4$, K$^{+}$(H$_2$O)$_6$ and \cli(H$_2$O)$_4$. To account for the experimental resolution and for the lifetime broadening due to the Auger decay of the core excited states, we convolved the theoretical spectra with a {\color{red} (Denis, exp. resolution? ) Voigt profile with a Gaussian of FWHM XX\,eV} and a Lorentzian of FWHM 0.74\,eV and 0.62\,eV in the case of \ki~and \cli, respectively \citep{Krause79:329}. We analyzed the core excited states by expanding the singly occupied natural orbitals (SONOs) $\psi_{i}$ of the microsolvated clusters in the basis of SONOs of the bare K$^{+}$ or \cli~ion $\chi_{nl}$
%
\begin{equation}\label{eq:sono_proj}
\psi_{i} = \sum_{nl} a^{i}_{nl} \chi_{nl}
\end{equation}
%
where $n$ and $l$ stand for the principal and orbital quantum numbers as described in Ref.\ \citep{miteva16:16671}. The expansion coefficients $a^{i}_{nl}$ show the degree of delocalization of the excited electron and the mixing of the core excited states in the crystal field created by the surrounding water molecules (see Figs.\ \ref{fg:knw_xas} and \ref{fg:clnw_xas}).


The final states following KLL resonant Auger decay of K$^{+}$ and \cli~were computed at the Configuration Interaction Singles (CIS) level using the Graphical Unitary Group Approach (GUGA) as implemented in the GAMESS-US package \citep{GUGA_PhysScr_21,GUGA_JCP_70,GUS}. We used the 6-311++G(2d,2p) basis set \citep{Blaudeau97:5016} augmented with 2s, 2p, 2d diffuse functions on \ki, and the cc-pVTZ basis set augmented with 6s, 6p, 6d diffuse KBJ functions \citep{Kaufmann89:2223} on \cli. The active space comprises the 2s and 2p orbitals of K/Cl with occupancy fixed to 6 and all virtual orbitals with occupancy fixed to 1. The remaining doubly occupied orbitals were frozen in the calculation. \citep{mosnier16:061401}