\section{Introduction}

XAS doesn't give information on the core-excited states

The information is contained in the combined XAS/AES spectrum

IP decreases faster than the energy of the 1st core-excited resonance

term value = IP - $E_{1st\,resonance}$ $\rightarrow$ decreases as the number of water molecules increases


The absorption of an X-ray photon by a chemical element leads to the promotion of a core-electron to empty orbitals or to the ionization continuum, which can be followed by an Auger relaxation on a femtosecond time scale or even shorter. The description of the corresponding core-excited or core-ionized states as well as of the states populated after the electronic decay provides extensive information both about the structure and the dynamics of the irradiated system (is very rich in information, both in terms of structure and dynamics of the irradiated system). These can be collected by spectroscopic methods consisting in measuring the kinetic energy of the emitted electron: Knowing the excitation energy, it is possible to determine precisely the binding energy of the ionized inner-shells, whose values are sensitive to the environment, i.e. the number and type of neighbors (electron donor or acceptor for instance). In terms of dynamics, the core-hole lifetime has been shown to be a very interesting  reference to probe ultrafast phenomena such as nuclear motion or charge transfer, both triggered by the photon absorption and occurring in particularly short time scale []. 


The Electron Spectroscopy for Chemical Analysis method (ESCA) has been extensively used in material science and atomic and molecular physics since its development in the 80’s for the reasons mentioned above. More recently, the pioneering works of M. Faubel and coworkers [] have given a new impulse since they succeeded to couple a microjet with an electron spectrometer, making it possible to track such electronic processes occurring in liquid media, especially in simple electrolyte aqueous solutions. In the last decade, a pretty large amount of new information has been collected based on this development [see e.g. references]. For instance, E.F. Aziz et al. \citep{aziz08:89} have shown that it is possible to excite the O1s orbital of OH$^-$ placed in aqueous solution, along a specific transition isolated from those of H$_2$O. The interpretation of the resonant Auger spectra coming from the relaxation of the oxygen 1s core-hole of hydroxide ion has allowed extracting information on its local environment, and on its rapid transport in water via a transition state accompanied by a mechanism of proton migration. B. Winter et al. \citep{Winter08:7130} have shown that in the case of Cl$^{-}$ ion in aqueous solution, the electronic states created by the interaction of the chlorine ion with the solvent can be populated by a resonant excitation of a Cl2p electron. These states are not visible in the  x-ray absorption spectrum, but are present in the resonant Auger spectrum. The analysis of the photon energy dependence of the energy position of the ``spectators'' states (i.e. the Cl2p electron promoted to the shared Cl/water electronic states do not participate in the relaxation) shows that they are neither completely localized nor completely delocalized but rather weakly localized around the solvated chlorine ion. The advantage of the core-orbital excitation technique is well demonstrated in this case since, in addition to the chemical selectivity, the lifetime of the core-hole gives a time base for the determination of the dynamical process: all these delocalization phenomena are almost instantaneous since they occur within the few fs of the Cl2p core-hole lifetime. The coupling chlorine - solvent is thus sufficient to allow an ultrafast charge transfer. In this article we present the results of a study focusing on the electronic structure of a KCl aqueous solution of 0.5M, excited in the vicinity of the K$^{+}$ and Cl$^{-}$ K-edges and probed by means of Auger and resonant Auger spectroscopy. After the absorption of an X-ray having an energy above the K-shell ionization thresholds, the targeted species is left with a core hole in its 1s shell and decays mostly by a KL$_{2,3}$L$_{2,3}$ Auger process, i.e. an electron from the 2p shell fills the 1s core-hole whereas a second 2p electron is ejected in the ionization continuum. For potassium and chlorine, the probability of a non-radiative decay (radiative decay) after a K-shell hole creation is about 90\% (10\%) \citep{Krause79:307}. Following this process, the electronic configuration of the final states is 2p$^{-2}$ leading to the $^1$D, $^1$S and $^3$P electronic states. In the case the 1s excited electron is not directly  ejected in the continuum but instead is promoted into an unoccupied orbital (np for a dipolar transition), then several relaxation schemes can occur. If the promoted 1s electron remains into the initially unoccupied orbital during the electronic decay, the process is called spectator Auger. The excited system 1s$^{-1}$np relaxes again mostly by involving two 2p electrons to reach the final configuration 2p$^{-2}$np. On the contrary, if at the time of the Auger decay the np electron is promoted to a n’p orbital with n’> n, then the process is called shake-up. Possibly, the np electron can fall down to an orbital n’p with n’<n, and in this case the process will be called shake-down. If the initially promoted electron is involved in the decay, the process is called participator Auger decay and the corresponding lines are shifted at higher kinetic energies compared with the spectator case. This channel will not be investigated in the present study. An additional possibility of electronic relaxation after creation of the 1s hole in potassium or chlorine ions involves neighboring water molecule. Such mechanism was for instance highlighted in a study on KCl dissolved in water and core ionized in the K$^{+}$ 2p shell \citep{Pokapanich09:7264}. The analysis of the potassium LMM Auger kinetic energy region by {\it ab initio} calculation shows that, in addition to the main lines, the extra structures located at higher kinetic energy can only be attributed to delocalized states involving both the potassium ion and water molecules. During the relaxation mechanism of the potassium 2p hole, one 3p electron of the same ion fills the core hole whereas an electron from the water valence shell is ejected in the ionization continuum. A previous study performed in the gas phase on argon (iso-electronic to Cl$^{-}$ and K$^{+}$) shows that the main relaxation channel of Ar1s excited to the first empty orbitals np ($n\geq4$), or directly ionized, leads mostly to the Ar$^+$[2p$^{-2}$]($^{1}$D$_{2}$)np or Ar$^{2+}$[2p$^{-2}$]($^{1}$D$_{2}$) final states \citep{ceolin15:022502}. The $^1$S and $^3$P states coming from the same 2p$^{-2}$ configuration were also measured but with a smaller intensity. The different mechanisms involved in the electronic decay (spectator Auger and shake processes) were clearly identified, and a formalism taking into account electronic lifetime interferences occurring in the core-excited state was used to explain the general shape of the pseudo partial cross sections 2p$^{-2}$(np) extracted from the measurements. The theory reproduces very well the experimental data and in particular it demonstrated that all the states with a 1s hole (i.e. with the excited 1s electron still bound or unbound to the system), must be coherently integrated in the model, to properly describe all the 2p$^{-2}$np and 2p$^{-2}$ pseudo cross sections. 


Thus, the main objective of this study is to collect information such as binding energy position, intensity, configuration, on the core-excited states involving a transition of the 1s orbital to either the first empty orbitals or into the ionization continuum, and to characterize the relaxation mechanisms, based on the electronic processes we highlighted for their isoelectronic counterpart in the gas phase. Potential delocalized electronic states between the solute (K$^{+}$/Cl$^{-}$) and the solvent (water) will also be probed. Our methods of investigation are the X-ray photoelectron spectroscopy and the (resonant) Auger spectroscopy in the hard X-ray range, and considering the high kinetic energy of the emitted electrons, this study is the first – to our knowledge – to clearly probe the bulk of the liquid, contrary to previous similar studies focusing on the liquid/vacuum interface. More details on the instrumental part are given in the next section.


We present {\it ab initio} calculations of the XAS spectra of microsolvated K$^{+}$(H$_2$O)$_n$ ($n = 1, 2, 4, 6$) clusters in order to interpret the experimental XAS and AES spectra. The calculations give further insight into the experimental results.

The paper is organized as follows. In the next section we present the experimental set-up and give a brief outline of the theoretical calculations. In Sec.\ \ref{sec:results} we present and discuss the results. The conclusions follow in Sec.\ \ref{sec:concl}.